\documentclass[12pt]{article}
\usepackage[T1]{fontenc}
\usepackage[T1]{polski}
\usepackage[utf8]{inputenc}
\newcommand{\BibTeX}{{\sc Bib}\TeX} 
\usepackage{graphicx}
\usepackage{amsfonts}
\usepackage{url}
\usepackage{babel,blindtext}


\setlength{\textheight}{21cm}

\title{{\bf Zadanie nr 2 - Próbkowanie i kwantyzacja}\linebreak
Cyfrowe Przetwarzanie Sygnałów}
\author{Julia Szymańska, 224441\and Przemysław Zdrzalik, 224466}
\date{21.04.2021r.}

\begin{document}
\clearpage\maketitle
\thispagestyle{empty}
\newpage
\setcounter{page}{1}


%%%%%%%%%%%%%%%%%%%%%%%%%%%%%%%%%%%%%%%%%%%%%%%%%%%%%%%%%%%%%%%%%%%%%%%%%%%%%

\section{Cel zadania}
Celem ćwiczenia jest budowa programu umożliwiającego wykonanie procesu konwersji analogowo-cyfrowej (A/C) i cyfrowo-analogowej (C/A) sygnałów. W programie dostępna jest:

\begin{itemize}
\item Konwersja A/C - próbkowanie równomierne
\item Konwersja A/C - kwantyzacja:

\begin{itemize}
\item Kwantyzacja równomierna z obcięciem
\item Kwantyzacja równomierna z zaokrąglaniem
\end{itemize}

\item Konwersja C/A - rekonstrukcja sygnału:
\begin{itemize}
\item Ekstrapolacja zerowego rzędu
\item Interpolacja pierwszego rzędu
\item Rekonstrukcja w oparciu o funkcję sinc
\end{itemize}
\ 

W programie możliwe jest również porównanie sygnału zrekonstruowanego z sygnałem orginalnym, w tym celu obliczane są cztery miary: 
\begin{itemize}
\item Błąd średniokwadratowy - MSE
\item Stosunek sygnał - szum - SNR
\item  Szczytowy stosunek sygnał - szum - PSNR
\item Maksymalna różnica - MD
\end{itemize}

\end{itemize}

%%%%%%%%%%%%%%%%%%%%%%%%%%%%%%%%%%%%%%%%%%%%%%%%%%%%%%%%%%%%%%%%%%%%%%%%%%%%%

\section{Wstęp teoretyczny}

W programie konwersje oraz miary do porównania sygnałów są obliczane na podstawie wzorów znajdujących sie w instrukcji do zadania drugeigo na platformie Wikamp \cite{dane}. 


%%%%%%%%%%%%%%%%%%%%%%%%%%%%%%%%%%%%%%%%%%%%%%%%%%%%%%%%%%%%%%%%%%%%%%%%%%%%%

\section{Eksperymenty i wyniki}

%%%%%%%%%%%%%%%%%%%%%%%%%%%%%%%%%%%%%%%%%

\subsection{Eksperyment nr 1 - Sygnał sinusoidalny Probkowanie}


W pierwszym eksperymencie analizujemy sygnał sinusoidalny. Wykonaliśmy próbkowanie na wygenerowanym sygnale, a następnie dokonaliśmy rekonstrukcji sygnału każdą z metod.

\begin{table}[h!]
\caption{Parametry wejściowe dla pierwszej wstępnej klasyfikacji. }
\centering
\vspace{0.1cm}
 \begin{tabular}{c c c c c}
    \textbf{Czas Początkowy} & \textbf{Czas Trwania}   & \textbf{Amplituda}  & \textbf{Okres} & \textbf{Częstotliwość Próbkowania}   \\
\hline
0 & 10 & 1 & 5 & 20 \\
\end {tabular}
\label {Parametry wejściowe dla pierwszego eksperymentu. }
\end{table}
\newpage


\begin{figure}[h!]
 \centering
 \includegraphics[width=14cm]{exp_1_sin_org.png}
 \vspace{-0.3cm}
 \caption{Oryginalny wygenerowany sygnał }
 \label{rysunek do eksperymentu 1 wariantu 1}
\end{figure}
\newpage

\subsubsection{Probkowanie}


\begin{figure}[h!]
 \centering
 \includegraphics[width=14cm]{exp_1_sin_prob.png}
 \vspace{-0.3cm}
 \caption{Wykres sygnału po wykonaniu próbkowania }
 \label{rysunek do eksperymentu 1 wariantu 2}
\end{figure}
\newpage

\begin{figure}[h!]
 \centering
 \includegraphics[width=14cm]{exp_1_sin_rek_1.png}
 \vspace{-0.3cm}
 \caption{Wykres sygnału funkcji sinusoidalenj po próbkowaniu i następnym zrekonstruowaniu metodą ekstrapolacji zerowego rzędu.}
 \label{rysunek do eksperymentu 1 wariantu 3}
\end{figure}
\newpage

\begin{figure}[h!]
 \centering
 \includegraphics[width=14cm]{exp_1_sin_rek_2.png}
 \vspace{-0.3cm}
 \caption{Wykres sygnału funkcji sinusoidalenj po próbkowaniu i następnym zrekonstruowaniu metodą interpolacją pierwszego rzędu.}
 \label{rysunek do eksperymentu 1 wariantu 4}
\end{figure}
\newpage


\begin{figure}[h!]
 \centering
 \includegraphics[width=14cm]{exp_1_sin_rek_3.png}
 \vspace{-0.3cm}
 \caption{Wykres sygnału funkcji sinusoidalenj po próbkowaniu i następnym zrekonstruowaniu metodą Rekonstrukcji w oparciu o funkcję sinc.}
 \label{rysunek do eksperymentu 1 wariantu 5}
\end{figure}
\newpage



\subsubsection{Kwantowanie}

\begin{figure}[h!]
 \centering
 \includegraphics[width=14cm]{exp_1_sin_prob.png}
 \vspace{-0.3cm}
 \caption{Wykres sygnału po wykonaniu próbkowania }
 \label{rysunek do eksperymentu 1 wariantu 2}
\end{figure}
\newpage



%%%%%%%%%%%%%%%%%%%%%%%%%%%%%%%%%%%%%%%%%

\subsection{Eksperyment nr 2 - Sygnał sinusoidalny Kwantyzacja}

W drugim eksperymencie analizujemy sygnał prostokątny. Wykonaliśmy kwantyzację na wygenerowanym sygnale, a następnie dokonaliśmy rekonstrukcji sygnału każdą z metod.

\begin{table}[h!]
\caption{Parametry wejściowe dla drugiej wstępnej klasyfikacji. }
\centering
\vspace{0.1cm}
 \begin{tabular}{c c c c c c}
    \textbf{Czas Początkowy} & \textbf{Czas Trwania}   & \textbf{Amplituda}  & \textbf{Okres} & \textbf{Częstotliwość Próbkowania} & \textbf {wsp. wyp.} \\
\hline
0 & 10 & 1 & 5 & 20 & 0.5 \\
\end {tabular}
\label {Parametry wejściowe dla drugiego eksperymentu. }
\end{table}



\begin{figure}[h!]
 \centering
 \includegraphics[width=14cm]{exp_1_prost_org.png}
 \vspace{-0.3cm}
 \caption{Oryginalny wygenerowany sygnał }
 \label{rysunek do eksperymentu 2 wariantu 1}
\end{figure}
\newpage




%%%%%%%%%%%%%%%%%%%%%%%%%%%%%%%%%%%%%%%%%%%%%%%%%%%%%%%%%%%%%%%%%%%%%%%%%%%%%

\section{Wnioski}

Zbudowany program umożliwia generację wybranych sygnałów, wyświetlenie ich parametrów, wykresów, a także histogramów, zapis/odczyt ich do/z plików, a także wykonanie operacji na dwóch zapisanych do plików sygnałach. Wygenerowane sygnały zgadzają się z przewidywanymi wynikami. Program działa poprawnie, cel zadania został osiągnięty. 


\begin{thebibliography}{0}
\bibitem{dane}  Wikamp, Instrukcja do zadania pierwszego, Dostępny w: \url{https://ftims.edu.p.lodz.pl/pluginfile.php/13449/mod_resource/content/0/zadanie2.pdf}

\end{thebibliography}



\end{document}