\documentclass[12pt]{article}
\usepackage[T1]{fontenc}
\usepackage[T1]{polski}
\usepackage[utf8]{inputenc}
\newcommand{\BibTeX}{{\sc Bib}\TeX} 
\usepackage{graphicx}
\usepackage{amsfonts}

\setlength{\textheight}{21cm}

\title{{\bf Zadanie nr 1 - Generacja sygnału i szumu}\linebreak
Cyfrowe Przetwarzanie Sygnałów}
\author{Julia Szymańska, 224441\and Przemysław Zdrzalik, 224466}
\date{26.03.2021r.}

\begin{document}
\clearpage\maketitle
\thispagestyle{empty}
\newpage
\setcounter{page}{1}

\section{Cel zadania}

Celem ćwiczenia jest wykonanie programu umożliwiającego generację wybranych sygnałów bądź impulsów, wyświetlenie ich parametrów, wykresów, a także histogramów, zapis/odczyt ich do/z plików, a także wykonanie operacji na dwóch zapisanych do plików sygnałach. 

\section{Wstęp teoretyczny}

Krótki opis wykorzystywanych metod~\cite{dowolna_etykieta_artykulu}. Proszę nie umieszczać ogólnie znanych z literatury
wzorów oraz definicji. Należy podać jaka metoda została zastosowana, dlaczego oraz podać wykorzystaną literaturę (korzystając z odwołań do pozycji bibliografii~\cite{dowolna_etykieta_ksiazki}).\\
Przygotowując bibliografię należy korzystać z podanego\\ 
szablonu \BibTeX-owego \texttt{bibliografia-wzor.bib}.


\section{Eksperymenty i wyniki}

%%%%%%%%%%%%%%%%%%%%%%%%%%%%%%%%%%%%%%%%%%%%%%%%%%%%%%%%%%%%%%%%%%%%%%%%%%%%%%%%%%%%%%%%%%%%%%%%%%%%%%%%%%%%%%%%%

\subsection{Eksperyment nr 1 - generowanie sygnału}

Eksperyment 1 polegał na wygenerowaniu sygnału sinusoidalnego wyprosotwanego jednopołówkowo na podstawie podanych parametrów oraz wyświetleniu wykresu i histogramu dla sygnału. 

\subsubsection{Założenia}

Funkcja opisująca sygnał: \begin{equation}  x(t) = \frac{1}{2}  A\{ \sin [ \frac{2\pi}{T}(t - t_1)] +  | \sin [ \frac{2 \pi}{T}(t - t_1)]  |\}  \end{equation}

\begin{figure}[h!]
 \centering
 \includegraphics[width=9.3cm]{spr_sygnal_sinus.png}
 \vspace{-0.3cm}
 \caption{Wykres przewidywany dla eksperymentu pierwszego.}
 \label{rysunek do eksperymentu pierwszego}
\end{figure}

\newpage


\subsubsection{Przebieg}

Do wygenerowania wykresu zostały podane poniższe parametry: 

\begin{figure}[h!]
 \centering
 \includegraphics[width=9.3cm]{par_1.png}
 \vspace{-0.3cm}
 \caption{Parametry dla eksperymentu pierwszego.}
 \label{rysunek do eksperymentu pierwszego}
\end{figure}

\newpage

Wygenerowany wykres sygnały sinusoidalnego wyprostowanego jednopołówkowo:


\begin{figure}[h!]
 \centering
 \includegraphics[width=9.3cm]{wykr_sygnal_sinus.png}
 \vspace{-0.3cm}
 \caption{Wykres wygenerowanego sygnału dla eksperymentu pierwszego.}
 \label{rysunek do eksperymentu pierwszego}
\end{figure}



Histogram dla wygenerowanego sygnału:

\begin{figure}[h!]
 \centering
 \includegraphics[width=9.3cm]{hist_1.png}
 \vspace{-0.3cm}
 \caption{Histogram dla wygenerowanego wykresu dla eksperymentu pierwszego.}
 \label{rysunek do eksperymentu pierwszego}
\end{figure}


\newpage

\subsubsection{Rezultat}

Dla wygenerowanego wykresu zostały obliczone jego parametry.  \ref{wyniki eksperymentu pierwszego}.

\begin{figure}[h!]
 \centering
 \includegraphics[width=9.3cm]{wyniki_1.png}
 \vspace{-0.3cm}
 \caption{Obliczone parametry wygenerowanego sygnału.}
 \label{rysunek do eksperymentu pierwszego}
\end{figure}

%%%%%%%%%%%%%%%%%%%%%%%%%%%%%%%%%%%%%%%%%%%%%%%%%%%%%%%%%%%%%%%%%%%%%%%%%%%%%%%%%%%%%%%%%%%%%%%%%%%%%%%%%%%%%%%%%

\subsection{Eksperyment nr 2 - generowanie impulsu}

Eksperyment 2 polegał na wygenerowaniu szumu impulsowego na podstawie podanych parametrów oraz wyświetleniu wykresu i histogramu dla szumu. 

\subsubsection{Założenia}

Szum impulsowy przyjmuje dwie wartości: wartość 0, wartość A - różna od zera. Przy czy wartość A występuje z zadanym p - prawdopodobieństem, podanym jako parametr. 

\begin{figure}[h!]
 \centering
 \includegraphics[width=9.3cm]{spr_2.png}
 \vspace{-0.3cm}
 \caption{Wykres przewidywany dla eksperymentu pierwszego.}
 \label{rysunek do eksperymentu pierwszego}
\end{figure}

\newpage


\subsubsection{Przebieg}

Do wygenerowania wykresu zostały podane poniższe parametry: 

\begin{figure}[h!]
 \centering
 \includegraphics[width=9.3cm]{par_2.png}
 \vspace{-0.3cm}
 \caption{Parametry dla eksperymentu pierwszego.}
 \label{rysunek do eksperymentu pierwszego}
\end{figure}
\

Wygenerowany wykres szumu impulsowego:


\begin{figure}[h!]
 \centering
 \includegraphics[width=9.3cm]{wykr_2.png}
 \vspace{-0.3cm}
 \caption{Wykres wygenerowanego szumu dla eksperymentu pierwszego.}
 \label{rysunek do eksperymentu pierwszego}
\end{figure}



Histogram dla wygenerowanego szumu:

\begin{figure}[h!]
 \centering
 \includegraphics[width=9.3cm]{hist_2.png}
 \vspace{-0.3cm}
 \caption{Histogram dla wygenerowanego szumu dla eksperymentu pierwszego.}
 \label{rysunek do eksperymentu pierwszego}
\end{figure}


\newpage

\subsubsection{Rezultat}

Dla wygenerowanego wykresu zostały obliczone jego parametry.  \ref{wyniki eksperymentu pierwszego}.

\begin{figure}[h!]
 \centering
 \includegraphics[width=9.3cm]{wyniki_1.png}
 \vspace{-0.3cm}
 \caption{Obliczone parametry wygenerowanego szumu.}
 \label{rysunek do eksperymentu pierwszego}
\end{figure}


\subsection{Eksperyment nr 3 - wykonanie operacji na sygnałach}

Eksperyment 3 polegał na dodaniu do siebie wcześniej wygenerowanego sygnału sinusoidalnego i szumu Gaussowsiego. 

\subsubsection{Założenia}

Szum impulsowy przyjmuje dwie wartości: wartość 0, wartość A - różna od zera. Przy czy wartość A występuje z zadanym p - prawdopodobieństem, podanym jako parametr. 

\begin{figure}[h!]
 \centering
 \includegraphics[width=9.3cm]{spr_2.png}
 \vspace{-0.3cm}
 \caption{Wykres przewidywany dla eksperymentu pierwszego.}
 \label{rysunek do eksperymentu pierwszego}
\end{figure}

\newpage


\subsubsection{Przebieg}

Do wygenerowania wykresu zostały podane poniższe parametry: 

\begin{figure}[h!]
 \centering
 \includegraphics[width=9.3cm]{par_2.png}
 \vspace{-0.3cm}
 \caption{Parametry dla eksperymentu pierwszego.}
 \label{rysunek do eksperymentu pierwszego}
\end{figure}
\

Wygenerowany wykres szumu impulsowego:


\begin{figure}[h!]
 \centering
 \includegraphics[width=9.3cm]{wykr_2.png}
 \vspace{-0.3cm}
 \caption{Wykres wygenerowanego szumu dla eksperymentu pierwszego.}
 \label{rysunek do eksperymentu pierwszego}
\end{figure}



Histogram dla wygenerowanego szumu:

\begin{figure}[h!]
 \centering
 \includegraphics[width=9.3cm]{hist_2.png}
 \vspace{-0.3cm}
 \caption{Histogram dla wygenerowanego szumu dla eksperymentu pierwszego.}
 \label{rysunek do eksperymentu pierwszego}
\end{figure}


\newpage

\subsubsection{Rezultat}

Dla wygenerowanego wykresu zostały obliczone jego parametry.  \ref{wyniki eksperymentu pierwszego}.

\begin{figure}[h!]
 \centering
 \includegraphics[width=9.3cm]{wyniki_1.png}
 \vspace{-0.3cm}
 \caption{Obliczone parametry wygenerowanego szumu.}
 \label{rysunek do eksperymentu pierwszego}
\end{figure}



\end{document}