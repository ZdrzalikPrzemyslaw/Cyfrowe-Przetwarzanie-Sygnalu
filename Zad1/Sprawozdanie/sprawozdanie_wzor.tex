\documentclass[12pt]{article}
\usepackage[T1]{fontenc}
\usepackage[T1]{polski}
\usepackage[utf8]{inputenc}
\newcommand{\BibTeX}{{\sc Bib}\TeX} 
\usepackage{graphicx}
\usepackage{amsfonts}

\setlength{\textheight}{21cm}

\title{{\bf Zadanie nr 1 - Generacja sygnału i szumu}\linebreak
Cyfrowe Przetwarzanie Sygnałów}
\author{Julia Szymańska, 224441\and Przemysław Zdrzalik, 224466}
\date{26.03.2021r.}

\begin{document}
\clearpage\maketitle
\thispagestyle{empty}
\newpage
\setcounter{page}{1}

\section{Cel zadania}

Celem ćwiczenia jest wykonanie programu umożliwiającego generację wybranych sygnałów bądź impulsów, wyświetlenie ich parametrów, wykresów, a także histogramów, zapis/odczyt ich do/z plików, a także wykonanie operacji na dwóch zapisanych do plików sygnałach. 

\section{Wstęp teoretyczny}



Krótki opis wykorzystywanych metod~\cite{dowolna_etykieta_artykulu}. Proszę nie umieszczać ogólnie znanych z literatury
wzorów oraz definicji. Należy podać jaka metoda została zastosowana, dlaczego oraz podać wykorzystaną literaturę (korzystając z odwołań do pozycji bibliografii~\cite{dowolna_etykieta_ksiazki}).\\
Przygotowując bibliografię należy korzystać z podanego\\ 
szablonu \BibTeX-owego \texttt{bibliografia-wzor.bib}.


\section{Eksperymenty i wyniki}

Opis wykonywanych eksperymentów. Wymagane jest ilustrowanie przeprowadzanych doświadczeń wykresami oraz tabelami.

\subsection{Eksperyment nr 1}

Eksperyment nr 1...\\
Identycznościowa funkcja aktywacji ma postać:
\begin{equation}
 \forall\: s \in \mathbb{R}\:\:\:\: f(s) = s
 \label{równanie funkcji identycznościowej}
\end{equation}
Jak widać z definicji (\ref{równanie funkcji identycznościowej}) funkcja ta...

\subsubsection{Założenia}

\subsubsection{Przebieg}
\newpage

\subsubsection{Rezultat}

Rezultaty badań eksperymentalnych przedstawione są w Tab. \ref{wyniki eksperymentu pierwszego}.
\begin{table}[h!]
 \caption{Rezultaty eksperymentu nr 1}
 \centering
 \vspace{0.2cm}
 \begin{tabular}{c c c c}
  \hline\hline\\[-0.4cm]
  \textbf{Przypadek} & \textbf{Metoda 1} & \textbf{Metoda 2} & \textbf{Metoda 3}\\[0.1cm]
  \hline
  \textbf{1} & 50 & 837 & 970  \\
  \textbf{2} & 47 & 877 & 230  \\
  \textbf{3} & 31 &  25 & 415  \\
  \textbf{4} & 35 & 144 & 2356 \\
  \textbf{5} & 45 & 300 & 556  \\ [0.1cm]
  \hline
 \end{tabular}
 \label{wyniki eksperymentu pierwszego}
\end{table}

\noindent Jak widać w Tab. \ref{wyniki eksperymentu pierwszego}...\newline
Graficzna interpretacja wyników z Tab. \ref{wyniki eksperymentu pierwszego} 
przedstawiona jest na wykresie Rys. \ref{rysunek do eksperymentu pierwszego} gdzie można zauważyć, że...
\begin{figure}[h!]
 \centering
 \includegraphics[width=9.3cm]{wykres.pdf}
 \vspace{-0.3cm}
 \caption{Wykres dla wyników eksperymentu pierwszego}
 \label{rysunek do eksperymentu pierwszego}
\end{figure}

\noindent Jak widać z wykresu Rys. \ref{rysunek do eksperymentu pierwszego}...\newline

\end{document}