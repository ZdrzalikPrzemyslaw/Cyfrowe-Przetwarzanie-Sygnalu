\documentclass[12pt]{article}
\usepackage[T1]{fontenc}
\usepackage[T1]{polski}
\usepackage[utf8]{inputenc}
\newcommand{\BibTeX}{{\sc Bib}\TeX} 
\usepackage{graphicx}
\usepackage{amsfonts}
\usepackage{url}
\usepackage{babel,blindtext}
\usepackage{hyperref}


\setlength{\textheight}{21cm}

\title{{\bf Zadanie nr 3 - Splot, filtracja i korelacja sygnałów}\linebreak
Cyfrowe Przetwarzanie Sygnałów}
\author{Julia Szymańska, 224441\and Przemysław Zdrzalik, 224466}
\date{19.05.2021r.}

\begin{document}
\clearpage\maketitle
\thispagestyle{empty}
\newpage
\setcounter{page}{1}


%%%%%%%%%%%%%%%%%%%%%%%%%%%%%%%%%%%%%%%%%%%%%%%%%%%%%%%%%%%%%%%%%%%%%%%%%%%%%

\section{Cel zadania}
Celem ćwiczenia jest budowa programu umożliwiającego wykonanie operacji splotu, korelacji bezpośredniej oraz korelacji na podstawie splotu. W programie możliwa jest również filtracja dolnoprzepustowa oraz środkowoprzepustowa z możliwośćią wyboru okna prostokątnego bądź Hamminga. 

%%%%%%%%%%%%%%%%%%%%%%%%%%%%%%%%%%%%%%%%%%%%%%%%%%%%%%%%%%%%%%%%%%%%%%%%%%%%%

\section{Wstęp teoretyczny}

 Do operacji splotu, korelacji oraz filtracji w programie zostały wykorzystane wzory znajdujące się w instrukcji do zadania trzeciego na platformie Wikamp \cite{dane}. 


%%%%%%%%%%%%%%%%%%%%%%%%%%%%%%%%%%%%%%%%%%%%%%%%%%%%%%%%%%%%%%%%%%%%%%%%%%%%%
\newpage
\section{Eksperymenty i wyniki}

%%%%%%%%%%%%%%%%%%%%%%%%%%%%%%%%%%%%%%%%%

\subsection{Eksperyment nr 1 - Operacja splotu}


W pierwszym eksperymencie analizujemy dwa sygnały sinusoidalne poddane operacji splotu. 

\begin{table}[h!]
\caption{Parametry wejściowe dla pierwszego sygnału sinusoidalnego.  }
\centering
\vspace{0.1cm}
 \begin{tabular}{c c c c c}
    \textbf{} & \textbf{Czas Rozpoczecia} & \textbf{Czas Zakonczenia}   & \textbf{Amplituda}  & \textbf{Okres}  \\
\hline
0 & 10 & 1 & 0.5 \\
\end {tabular}
\label {Parametry wejściowe dla pierwszego eksperymetnu. }
\end{table}

\begin{figure}[h!]
 \centering
 \includegraphics[width=14cm]{sinus.png}
 \vspace{-0.3cm}
 \caption{Oryginalny wygenerowany sygnał sinusoidalny. }
 \label{rysunek do eksperymentu 1 wariantu 1}
\end{figure}
\newpage

\begin{table}[h!]
\caption{Parametry wejściowe dla drugiego sygnału sinusoidalnego.  }
\centering
\vspace{0.1cm}
 \begin{tabular}{c c c c c}
    \textbf{} & \textbf{Czas Rozpoczecia} & \textbf{Czas Zakonczenia}   & \textbf{Amplituda}  & \textbf{Okres}  \\
\hline
0 & 10 & 4 & 5 \\
\end {tabular}
\label {Parametry wejściowe dla trzeciego eksperymetnu. }
\end{table}
\newpage

\begin{figure}[h!]
 \centering
 \includegraphics[width=14cm]{sinus_2.png}
 \vspace{-0.3cm}
 \caption{Oryginalny wygenerowany drugi sygnał sinusoidalny. }
 \label{rysunek do eksperymentu 1 wariantu 1}
\end{figure}
\newpage

Wygenerowany wykres operacji splotu dla sygnału sinusoidalnego. 

\begin{figure}[h!]
 \centering
 \includegraphics[width=14cm]{sinus_splot.png}
 \vspace{-0.3cm}
 \caption{Wynik operacji splotu dwóch sygnałów sinusoidalnych. }
 \label{rysunek do eksperymentu 1 wariantu 1}
\end{figure}
\newpage

\begin{figure}[h!]
 \centering
 \includegraphics[width=14cm]{sinus_splot_hist.png}
 \vspace{-0.3cm}
 \caption{Histogram dla wyniku operacji splotu dwóch sygnałów sinusoidalnych. }
 \label{rysunek do eksperymentu 1 wariantu 1}
\end{figure}
\newpage


%%%%%%%%%%%%%%%%%%%%%%%%%%%%%%%%%%%%%%%%
\subsection{Eksperyment nr 2 - Operacja splotu dal różnych typów sygnałów}


W drugim eksperymencie analizujemy dwa sygnały, sygnał prostokątny oraz szum gaussowski poddane operacji splotu. 

\begin{table}[h!]
\caption{Parametry wejściowe dla sygnału prostokątnego.  }
\centering
\vspace{0.1cm}
 \begin{tabular}{c c c c c}
    \textbf{} & \textbf{Czas Rozpoczecia} & \textbf{Czas Zakonczenia}   & \textbf{Amplituda}  & \textbf{Okres}  \\
\hline
0 & 10 & 1 & 1 \\
\end {tabular}
\label {Parametry wejściowe dla durgiego eksperymetnu. }
\end{table}

\begin{figure}[h!]
 \centering
 \includegraphics[width=14cm]{prostok.png}
 \vspace{-0.3cm}
 \caption{Oryginalny wygenerowany sygnał prostokątny. }
 \label{rysunek do eksperymentu 1 wariantu 1}
\end{figure}
\newpage

\begin{table}[h!]
\caption{Parametry wejściowe dla szumu gaussowskiego.  }
\centering
\vspace{0.1cm}
 \begin{tabular}{c c c c c}
    \textbf{} & \textbf{Czas Rozpoczecia} & \textbf{Czas Zakonczenia}   & \textbf{Amplituda}   \\
\hline
0 & 10 & 3 \\
\end {tabular}
\label {Parametry wejściowe dla trzeciego eksperymetnu. }
\end{table}
\newpage

\begin{figure}[h!]
 \centering
 \includegraphics[width=14cm]{gauss.png}
 \vspace{-0.3cm}
 \caption{Oryginalny wygenerowany sygnał gaussowski. }
 \label{rysunek do eksperymentu 1 wariantu 1}
\end{figure}
\newpage

Wygenerowany wykres operacji splotu dla sygnału prostokątnego i szumu gaussowskiego. 

\begin{figure}[h!]
 \centering
 \includegraphics[width=14cm]{gauss_splot.png}
 \vspace{-0.3cm}
 \caption{Wynik operacji splotu sygnału prostokątnego i szumu gaussowskiego. 
 }
 \label{rysunek do eksperymentu 1 wariantu 1}
\end{figure}
\newpage

\begin{figure}[h!]
 \centering
 \includegraphics[width=14cm]{gauss_splot_hist.png}
 \vspace{-0.3cm}
 \caption{Histogram dla wyniku operacji splotu sygnału prostokątnego i szumu gaussowskiego. 
 }
 \label{rysunek do eksperymentu 1 wariantu 1}
\end{figure}
\newpage




%%%%%%%%%%%%%%%%%%%%%%%%%%%%%%%%%%%%%%%%

\subsection{Eksperyment nr 3 - Operacja splotu dal różnych typów sygnałów}

W trzecim eksperymencie analizujemy dwa sygnały, sygnał trójkątny oraz sygnał sinusoidalny wyprostowany jednopołówkowo poddane operacji splotu. 

\begin{table}[h!]
\caption{Parametry wejściowe dla sygnału trójkątnego.  }
\centering
\vspace{0.1cm}
 \begin{tabular}{c c c c c}
     \textbf{Czas Rozpoczecia} & \textbf{Czas Zakonczenia}   & \textbf{Amplituda}  & \textbf{Okres} & \textbf{Współczynik wypełnienia}  \\
\hline
0 & 10 & 6 & 2 & 0.5 \\
\end {tabular}
\label {Parametry wejściowe dla trzeciego eksperymetnu. }
\end{table}

\begin{figure}[h!]
 \centering
 \includegraphics[width=14cm]{trojkat.png}
 \vspace{-0.3cm}
 \caption{Oryginalny wygenerowany sygnał trójkątny. }
 \label{rysunek do eksperymentu 1 wariantu 51}
\end{figure}
\newpage

\begin{table}[h!]
\caption{Parametry wejściowe dla szumu sinusoidalnego wyprostowanego jednopołówkowo.  }
\centering
\vspace{0.1cm}
 \begin{tabular}{c c c c c }
    \textbf{Czas Rozpoczecia} & \textbf{Czas Zakonczenia}   & \textbf{Amplituda} &   & \textbf{Okres} \\
\hline
0 & 10 & 1 & 0.4 \\
\end {tabular}
\label {Parametry wejściowe dla trzeciego eksperymetnu. }
\end{table}
\newpage

\begin{figure}[h!]
 \centering
 \includegraphics[width=14cm]{jedno.png}
 \vspace{-0.3cm}
 \caption{Oryginalny wygenerowany sygnał sinusoidalny wyprostowany jednopołówkowo. }
 \label{rysunek do eksperymentu 1 wariantu 4}
\end{figure}
\newpage

Wygenerowany wykres operacji splotu dla sygnału trójkątnego oraz sygnału sinusoidalnego wyprostowanego jednopołówkowo.

\begin{figure}[h!]
 \centering
 \includegraphics[width=14cm]{jedno_splot.png}
 \vspace{-0.3cm}
 \caption{Wynik operacji splotu sygnału trójkątnego oraz sygnału sinusoidalnego wyprostowanego jednopołówkowo.}
 \label{rysunek do eksperymentu 1 wariantu 5}
\end{figure}
\newpage


\begin{figure}[h!]
 \centering
 \includegraphics[width=14cm]{jedno_splot_hist.png}
 \vspace{-0.3cm}
 \caption{Histogram dla wyniku operacji splotu sygnału trójkątnego oraz sygnału sinusoidalnego wyprostowanego jednopołówkowo.}
 \label{rysunek do eksperymentu 1 wariantu 1}
\end{figure}
\newpage


%%%%%%%%%%%%%%%%%%%%%%%%%%%%%%%%%%%%%%%%

\subsection{Eksperyment nr 4 - Koleracja bezpośredniej dla dwóch sygnałów sinusoidalnych}

W czwartym eksperymencie analizujemy dwa sygnały sinusoidalne poddane operacji korelacji bezpośredniej. 

\begin{table}[h!]
\caption{Parametry wejściowe dla pierwszego sygnału sinusoidalnego.  }
\centering
\vspace{0.1cm}
 \begin{tabular}{c c c c c}
    \textbf{} & \textbf{Czas Rozpoczecia} & \textbf{Czas Zakonczenia}   & \textbf{Amplituda}  & \textbf{Okres}  \\
\hline
0 & 10 & 1 & 0.5 \\
\end {tabular}
\label {Parametry wejściowe dla pierwszego eksperymetnu. }
\end{table}


\begin{figure}[h!]
 \centering
 \includegraphics[width=14cm]{sinus.png}
 \vspace{-0.3cm}
 \caption{Oryginalny wygenerowany sygnał sinusoidalny. }
 \label{rysunek do eksperymentu 1 wariantu 1}
\end{figure}
\newpage

\begin{table}[h!]
\caption{Parametry wejściowe dla drugiego sygnału sinusoidalnego.  }
\centering
\vspace{0.1cm}
 \begin{tabular}{c c c c c}
    \textbf{} & \textbf{Czas Rozpoczecia} & \textbf{Czas Zakonczenia}   & \textbf{Amplituda}  & \textbf{Okres}  \\
\hline
0 & 10 & 4 & 5 \\
\end {tabular}
\label {Parametry wejściowe dla trzeciego eksperymetnu. }
\end{table}
\newpage

\begin{figure}[h!]
 \centering
 \includegraphics[width=14cm]{sinus_2.png}
 \vspace{-0.3cm}
 \caption{Oryginalny wygenerowany drugi sygnał sinusoidalny. }
 \label{rysunek do eksperymentu 1 wariantu 1}
\end{figure}
\newpage

Wygenerowany wykres operacji korelacji bezpośredniej dla sygnału sinusoidalnego. 

\begin{figure}[h!]
 \centering
 \includegraphics[width=14cm]{sinus_kor_bez.png}
 \vspace{-0.3cm}
 \caption{Wynik operacji korelacji bezpośredniej dwóch sygnałów sinusoidalnych. }
 \label{rysunek do eksperymentu 1 wariantu 1}
\end{figure}
\newpage

\begin{figure}[h!]
 \centering
 \includegraphics[width=14cm]{sinus_kor_bez-hist.png}
 \vspace{-0.3cm}
 \caption{Histogram dla wyniku operacji korelacji bezpośredniej dwóch sygnałów sinusoidalnych. }
 \label{rysunek do eksperymentu 1 wariantu 1}
\end{figure}
\newpage

%%%%%%%%%%%%%%%%%%%%%%%%%%%%%%%%%%%%%%%%

\subsection{Eksperyment nr 5 - Koleracja dla dwóch sygnałów sinusoidalnych}

W piątym eksperymencie analizujemy dwa sygnały sinusoidalne poddane operacji korelacji z użyciem splotu. 

\begin{table}[h!]
\caption{Parametry wejściowe dla pierwszego sygnału sinusoidalnego.  }
\centering
\vspace{0.1cm}
 \begin{tabular}{c c c c c}
    \textbf{} & \textbf{Czas Rozpoczecia} & \textbf{Czas Zakonczenia}   & \textbf{Amplituda}  & \textbf{Okres}  \\
\hline
0 & 10 & 1 & 0.5 \\
\end {tabular}
\label {Parametry wejściowe dla pierwszego eksperymetnu. }
\end{table}

\begin{figure}[h!]
 \centering
 \includegraphics[width=14cm]{sinus.png}
 \vspace{-0.3cm}
 \caption{Oryginalny wygenerowany sygnał sinusoidalny. }
 \label{rysunek do eksperymentu 1 wariantu 1}
\end{figure}
\newpage

\begin{table}[h!]
\caption{Parametry wejściowe dla drugiego sygnału sinusoidalnego.  }
\centering
\vspace{0.1cm}
 \begin{tabular}{c c c c c}
    \textbf{} & \textbf{Czas Rozpoczecia} & \textbf{Czas Zakonczenia}   & \textbf{Amplituda}  & \textbf{Okres}  \\
\hline
0 & 10 & 4 & 5 \\
\end {tabular}
\label {Parametry wejściowe dla trzeciego eksperymetnu. }
\end{table}
\newpage

\begin{figure}[h!]
 \centering
 \includegraphics[width=14cm]{sinus_2.png}
 \vspace{-0.3cm}
 \caption{Oryginalny wygenerowany drugi sygnał sinusoidalny. }
 \label{rysunek do eksperymentu 1 wariantu 1}
\end{figure}
\newpage

Wygenerowany wykres operacji z użyciem splotu korelacji dla sygnału sinusoidalnego. 

\begin{figure}[h!]
 \centering
 \includegraphics[width=14cm]{sinus_kor_bez.png}
 \vspace{-0.3cm}
 \caption{Wynik operacji korelacji z użyciem splotu dwóch sygnałów sinusoidalnych. }
 \label{rysunek do eksperymentu 1 wariantu 1}
\end{figure}
\newpage

\begin{figure}[h!]
 \centering
 \includegraphics[width=14cm]{sinus_kor_bez-hist.png}
 \vspace{-0.3cm}
 \caption{Histogram dla wyniku operacji korelacji z użyciem splotu dwóch sygnałów sinusoidalnych. }
 \label{rysunek do eksperymentu 1 wariantu 1}
\end{figure}
\newpage


%%%%%%%%%%%%%%%%%%%%%%%%%%%%%%%%%%%%%%%%
\subsection{Eksperyment nr 6 - Operacja korelacji bezpośredniej dla różnych typów sygnałów}


W szóstym eksperymencie analizujemy dwa sygnały, sygnał prostokątny oraz szum gaussowski poddane operacji korelacji bezpośredniej . 

\begin{table}[h!]
\caption{Parametry wejściowe dla sygnału prostokątnego.  }
\centering
\vspace{0.1cm}
 \begin{tabular}{c c c c c}
    \textbf{} & \textbf{Czas Rozpoczecia} & \textbf{Czas Zakonczenia}   & \textbf{Amplituda}  & \textbf{Okres}  \\
\hline
0 & 10 & 1 & 1 \\
\end {tabular}
\label {Parametry wejściowe dla durgiego eksperymetnu. }
\end{table}

\begin{figure}[h!]
 \centering
 \includegraphics[width=14cm]{prostok.png}
 \vspace{-0.3cm}
 \caption{Oryginalny wygenerowany sygnał prostokątny. }
 \label{rysunek do eksperymentu 1 wariantu 1}
\end{figure}
\newpage

\begin{table}[h!]
\caption{Parametry wejściowe dla szumu gaussowskiego.  }
\centering
\vspace{0.1cm}
 \begin{tabular}{c c c c c}
    \textbf{} & \textbf{Czas Rozpoczecia} & \textbf{Czas Zakonczenia}   & \textbf{Amplituda}   \\
\hline
0 & 10 & 3 \\
\end {tabular}
\label {Parametry wejściowe dla trzeciego eksperymetnu. }
\end{table}
\newpage

\begin{figure}[h!]
 \centering
 \includegraphics[width=14cm]{gauss.png}
 \vspace{-0.3cm}
 \caption{Oryginalny wygenerowany sygnał gaussowski. }
 \label{rysunek do eksperymentu 1 wariantu 1}
\end{figure}
\newpage

Wygenerowany wykres operacji korelacji bezpośredniej  dla sygnału prostokątnego i szumu gaussowskiego. 

\begin{figure}[h!]
 \centering
 \includegraphics[width=14cm]{gauss_kor_2.png}
 \vspace{-0.3cm}
 \caption{Wynik operacji korelacji bezpośredniej sygnału prostokątnego i szumu gaussowskiego. 
 }
 \label{rysunek do eksperymentu 1 wariantu 1}
\end{figure}
\newpage

\begin{figure}[h!]
 \centering
 \includegraphics[width=14cm]{gauss_kor_hist_2.png}
 \vspace{-0.3cm}
 \caption{Histogram dla wyniku operacji korelacji bezpośredniej sygnału prostokątnego i szumu gaussowskiego. 
 }
 \label{rysunek do eksperymentu 1 wariantu 1}
\end{figure}
\newpage


%%%%%%%%%%%%%%%%%%%%%%%%%%%%%%%%%%%%%%%%
\subsection{Eksperyment nr 7 - Operacja korelacji z użyciem splotu dla różnych typów sygnałów}


W siódmym eksperymencie analizujemy dwa sygnały, sygnał prostokątny oraz szum gaussowski poddane operacji korelacji z użyciem splotu . 

\begin{table}[h!]
\caption{Parametry wejściowe dla sygnału prostokątnego.  }
\centering
\vspace{0.1cm}
 \begin{tabular}{c c c c c}
    \textbf{} & \textbf{Czas Rozpoczecia} & \textbf{Czas Zakonczenia}   & \textbf{Amplituda}  & \textbf{Okres}  \\
\hline
0 & 10 & 1 & 1 \\
\end {tabular}
\label {Parametry wejściowe dla durgiego eksperymetnu. }
\end{table}

\begin{figure}[h!]
 \centering
 \includegraphics[width=14cm]{prostok.png}
 \vspace{-0.3cm}
 \caption{Oryginalny wygenerowany sygnał prostokątny. }
 \label{rysunek do eksperymentu 1 wariantu 1}
\end{figure}
\newpage

\begin{table}[h!]
\caption{Parametry wejściowe dla szumu gaussowskiego.  }
\centering
\vspace{0.1cm}
 \begin{tabular}{c c c c c}
    \textbf{} & \textbf{Czas Rozpoczecia} & \textbf{Czas Zakonczenia}   & \textbf{Amplituda}   \\
\hline
0 & 10 & 3 \\
\end {tabular}
\label {Parametry wejściowe dla trzeciego eksperymetnu. }
\end{table}
\newpage

\begin{figure}[h!]
 \centering
 \includegraphics[width=14cm]{gauss.png}
 \vspace{-0.3cm}
 \caption{Oryginalny wygenerowany sygnał gaussowski. }
 \label{rysunek do eksperymentu 1 wariantu 1}
\end{figure}
\newpage

Wygenerowany wykres operacji korelacji z użyciem splotu  dla sygnału prostokątnego i szumu gaussowskiego. 

\begin{figure}[h!]
 \centering
 \includegraphics[width=14cm]{gauss_kor_2.png}
 \vspace{-0.3cm}
 \caption{Wynik operacji korelacji z użyciem splotu sygnału prostokątnego i szumu gaussowskiego. 
 }
 \label{rysunek do eksperymentu 1 wariantu 1}
\end{figure}
\newpage

\begin{figure}[h!]
 \centering
 \includegraphics[width=14cm]{gauss_kor_hist_2.png}
 \vspace{-0.3cm}
 \caption{Histogram dla wyniku operacji korelacji bezpośredniej sygnału prostokątnego i szumu gaussowskiego. 
 }
 \label{rysunek do eksperymentu 1 wariantu 1}
\end{figure}
\newpage


%%%%%%%%%%%%%%%%%%%%%%%%%%%%%%%%%%%%%%%%

\subsection{Eksperyment nr 8 - Operacja korelacji bezpośredniej dla różnych typów sygnałów}

W ósmym eksperymencie analizujemy dwa sygnały, sygnał trójkątny oraz sygnał sinusoidalny wyprostowany jednopołówkowo poddane operacji korelacji bezpośredniej. 

\begin{table}[h!]
\caption{Parametry wejściowe dla sygnału trójkątnego.  }
\centering
\vspace{0.1cm}
 \begin{tabular}{c c c c c}
     \textbf{Czas Rozpoczecia} & \textbf{Czas Zakonczenia}   & \textbf{Amplituda}  & \textbf{Okres} & \textbf{Współczynik wypełnienia}  \\
\hline
0 & 10 & 6 & 2 & 0.5 \\
\end {tabular}
\label {Parametry wejściowe dla trzeciego eksperymetnu. }
\end{table}

\begin{figure}[h!]
 \centering
 \includegraphics[width=14cm]{trojkat.png}
 \vspace{-0.3cm}
 \caption{Oryginalny wygenerowany sygnał trójkątny. }
 \label{rysunek do eksperymentu 1 wariantu 51}
\end{figure}
\newpage

\begin{table}[h!]
\caption{Parametry wejściowe dla szumu sinusoidalnego wyprostowanego jednopołówkowo.  }
\centering
\vspace{0.1cm}
 \begin{tabular}{c c c c c }
    \textbf{Czas Rozpoczecia} & \textbf{Czas Zakonczenia}   & \textbf{Amplituda} &   & \textbf{Okres} \\
\hline
0 & 10 & 1 & 0.4 \\
\end {tabular}
\label {Parametry wejściowe dla trzeciego eksperymetnu. }
\end{table}
\newpage

\begin{figure}[h!]
 \centering
 \includegraphics[width=14cm]{jedno.png}
 \vspace{-0.3cm}
 \caption{Oryginalny wygenerowany sygnał sinusoidalny wyprostowany jednopołówkowo. }
 \label{rysunek do eksperymentu 1 wariantu 4}
\end{figure}
\newpage

Wygenerowany wykres operacji korelacji bezpośredniej dla sygnału trójkątnego oraz sygnału sinusoidalnego wyprostowanego jednopołówkowo.

\begin{figure}[h!]
 \centering
 \includegraphics[width=14cm]{jedno_kor_bez.png}
 \vspace{-0.3cm}
 \caption{Wynik operacji korelacji bezpośredniej sygnału trójkątnego oraz sygnału sinusoidalnego wyprostowanego jednopołówkowo.}
 \label{rysunek do eksperymentu 1 wariantu 5}
\end{figure}
\newpage


\begin{figure}[h!]
 \centering
 \includegraphics[width=14cm]{jedno_kor_bez_hist.png}
 \vspace{-0.3cm}
 \caption{Histogram dla wyniku operacji korelacji bezpośredniej sygnału trójkątnego oraz sygnału sinusoidalnego wyprostowanego jednopołówkowo.}
 \label{rysunek do eksperymentu 1 wariantu 1}
\end{figure}
\newpage




%%%%%%%%%%%%%%%%%%%%%%%%%%%%%%%%%%%%%%%%

\subsection{Eksperyment nr 9 - Operacja korelacji z użyciem splotu dla różnych typów sygnałów}

W dziwiątym eksperymencie analizujemy dwa sygnały, sygnał trójkątny oraz sygnał sinusoidalny wyprostowany jednopołówkowo poddane operacji korelacji z użyciem splotu. 

\begin{table}[h!]
\caption{Parametry wejściowe dla sygnału trójkątnego.  }
\centering
\vspace{0.1cm}
 \begin{tabular}{c c c c c}
     \textbf{Czas Rozpoczecia} & \textbf{Czas Zakonczenia}   & \textbf{Amplituda}  & \textbf{Okres} & \textbf{Współczynik wypełnienia}  \\
\hline
0 & 10 & 6 & 2 & 0.5 \\
\end {tabular}
\label {Parametry wejściowe dla trzeciego eksperymetnu. }
\end{table}

\begin{figure}[h!]
 \centering
 \includegraphics[width=14cm]{trojkat.png}
 \vspace{-0.3cm}
 \caption{Oryginalny wygenerowany sygnał trójkątny. }
 \label{rysunek do eksperymentu 1 wariantu 51}
\end{figure}
\newpage

\begin{table}[h!]
\caption{Parametry wejściowe dla szumu sinusoidalnego wyprostowanego jednopołówkowo.  }
\centering
\vspace{0.1cm}
 \begin{tabular}{c c c c c }
    \textbf{Czas Rozpoczecia} & \textbf{Czas Zakonczenia}   & \textbf{Amplituda} &   & \textbf{Okres} \\
\hline
0 & 10 & 1 & 0.4 \\
\end {tabular}
\label {Parametry wejściowe dla trzeciego eksperymetnu. }
\end{table}
\newpage

\begin{figure}[h!]
 \centering
 \includegraphics[width=14cm]{jedno.png}
 \vspace{-0.3cm}
 \caption{Oryginalny wygenerowany sygnał sinusoidalny wyprostowany jednopołówkowo. }
 \label{rysunek do eksperymentu 1 wariantu 4}
\end{figure}
\newpage

Wygenerowany wykres operacji korelacji z użyciem splotu dla sygnału trójkątnego oraz sygnału sinusoidalnego wyprostowanego jednopołówkowo.

\begin{figure}[h!]
 \centering
 \includegraphics[width=14cm]{jedno_kor_bez.png}
 \vspace{-0.3cm}
 \caption{Wynik operacji korelacji z użyciem splotu sygnału trójkątnego oraz sygnału sinusoidalnego wyprostowanego jednopołówkowo.}
 \label{rysunek do eksperymentu 1 wariantu 5}
\end{figure}
\newpage


\begin{figure}[h!]
 \centering
 \includegraphics[width=14cm]{jedno_kor_bez_hist.png}
 \vspace{-0.3cm}
 \caption{Histogram dla wyniku operacji korelacji z użyciem splotu sygnału trójkątnego oraz sygnału sinusoidalnego wyprostowanego jednopołówkowo.}
 \label{rysunek do eksperymentu 1 wariantu 1}
\end{figure}
\newpage


%%%%%%%%%%%%%%%%%%%%%%%%%%%%%%%%%%%%%%%%

\subsection{Eksperyment nr 10 - Operacja filtracji dolnoprzepustowej z oknem prostokątnym}

W dziesiątym eksperymencie generujemy filtrację dolnoprzepustową z oknem prostokątnym.

\begin{table}[h!]
\caption{Parametry wejściowe filtracji dolnoprzepustowej z oknem prostokątnym.  }
\centering
\vspace{0.1cm}
 \begin{tabular}{c c c}
     \textbf{Częstotliwość próbkowania} & \textbf{Liczba próbek}   & \textbf{Częstosliwość odcięcia} \\
\hline
10 & 29 & 3 \\
\end {tabular}
\label {Parametry wejściowe dla trzeciego eksperymetnu. }
\end{table}

Wygenerowany wykres filtracji dolnoprzepustowej z oknem prostokątnym.

\begin{figure}[h!]
 \centering
 \includegraphics[width=14cm]{filtr_dol_n.png}
 \vspace{-0.3cm}
 \caption{Wynik filtracji dolnoprzepustowej z oknem prostokątnym. }
 \label{rysunek do eksperymentu 1 wariantu 5}
\end{figure}
\newpage


\begin{figure}[h!]
 \centering
 \includegraphics[width=14cm]{filtr_dol_n_hist.png}
 \vspace{-0.3cm}
 \caption{Histogram dla wyniku filtracji dolnoprzepustowej z oknem prostokątnym. }
 \label{rysunek do eksperymentu 1 wariantu 1}
\end{figure}
\newpage


%%%%%%%%%%%%%%%%%%%%%%%%%%%%%%%%%%%%%%%%

\subsection{Eksperyment nr 11 - Operacja filtracji dolnoprzepustowej z oknem Hamminga}

W jedenastym eksperymencie generujemy filtrację dolnoprzepustową z oknem Hammina.

\begin{table}[h!]
\caption{Parametry wejściowe filtracji dolnoprzepustowej z oknem Hamminga.  }
\centering
\vspace{0.1cm}
 \begin{tabular}{c c c}
     \textbf{Częstotliwość próbkowania} & \textbf{Liczba próbek}   & \textbf{Częstosliwość odcięcia} \\
\hline
10 & 29 & 3 \\
\end {tabular}
\label {Parametry wejściowe dla trzeciego eksperymetnu. }
\end{table}

Wygenerowany wykres filtracji dolnoprzepustowej z oknem Hamminga.

\begin{figure}[h!]
 \centering
 \includegraphics[width=14cm]{filtr_dol_n_ham.png}
 \vspace{-0.3cm}
 \caption{Wynik filtracji dolnoprzepustowej z oknem Hamminga. }
 \label{rysunek do eksperymentu 1 wariantu 5}
\end{figure}
\newpage


\begin{figure}[h!]
 \centering
 \includegraphics[width=14cm]{filtr_dol_n_ham_hist.png}
 \vspace{-0.3cm}
 \caption{Histogram dla wyniku filtracji dolnoprzepustowej z oknem Hamminga. }
 \label{rysunek do eksperymentu 1 wariantu 1}
\end{figure}
\newpage



%%%%%%%%%%%%%%%%%%%%%%%%%%%%%%%%%%%%%%%%

\subsection{Eksperyment nr 12 - Operacja filtracji środkowoprzepustowa z oknem prostokątnym}

W dziesiątym eksperymencie generujemy filtrację środkowoprzepustowa z oknem prostokątnym.

\begin{table}[h!]
\caption{Parametry wejściowe filtracji środkowoprzepustowa z oknem prostokątnym.  }
\centering
\vspace{0.1cm}
 \begin{tabular}{c c c}
     \textbf{Częstotliwość próbkowania} & \textbf{Liczba próbek}   & \textbf{Częstosliwość odcięcia} \\
\hline
10 & 29 & 3 \\
\end {tabular}
\label {Parametry wejściowe dla trzeciego eksperymetnu. }
\end{table}

Wygenerowany wykres filtracji środkowoprzepustowa z oknem prostokątnym.

\begin{figure}[h!]
 \centering
 \includegraphics[width=14cm]{filtr_srod_n.png}
 \vspace{-0.3cm}
 \caption{Wynik filtracji środkowoprzepustowa z oknem prostokątnym. }
 \label{rysunek do eksperymentu 1 wariantu 5}
\end{figure}
\newpage


\begin{figure}[h!]
 \centering
 \includegraphics[width=14cm]{filtr_srod_n_hist.png}
 \vspace{-0.3cm}
 \caption{Histogram dla wyniku filtracji środkowoprzepustowa z oknem prostokątnym. }
 \label{rysunek do eksperymentu 1 wariantu 1}
\end{figure}
\newpage


%%%%%%%%%%%%%%%%%%%%%%%%%%%%%%%%%%%%%%%%

\subsection{Eksperyment nr 13 - Operacja filtracji środkowoprzepustowa z oknem Hamminga}

W jedenastym eksperymencie generujemy filtrację środkowoprzepustowa z oknem Hammina.

\begin{table}[h!]
\caption{Parametry wejściowe filtracji środkowoprzepustowa z oknem Hamminga.  }
\centering
\vspace{0.1cm}
 \begin{tabular}{c c c}
     \textbf{Częstotliwość próbkowania} & \textbf{Liczba próbek}   & \textbf{Częstosliwość odcięcia} \\
\hline
10 & 29 & 3 \\
\end {tabular}
\label {Parametry wejściowe dla trzeciego eksperymetnu. }
\end{table}

Wygenerowany wykres filtracji środkowoprzepustowa z oknem Hamminga.

\begin{figure}[h!]
 \centering
 \includegraphics[width=14cm]{filtr_srod_n_ham.png}
 \vspace{-0.3cm}
 \caption{Wynik filtracji środkowoprzepustowa z oknem Hamminga. }
 \label{rysunek do eksperymentu 1 wariantu 5}
\end{figure}
\newpage


\begin{figure}[h!]
 \centering
 \includegraphics[width=14cm]{filtr_srod_n_ham_hist.png}
 \vspace{-0.3cm}
 \caption{Histogram dla wyniku filtracji dolnoprzepustowej z oknem Hamminga. }
 \label{rysunek do eksperymentu 1 wariantu 1}
\end{figure}
\newpage






%%%%%%%%%%%%%%%%%%%%%%%%%%%%%%%%%%%%%%%%%%%%%%%%%%%%%%%%%%%%%%%%%%%%%%%%%%%%%
\newpage
\section{Wnioski}

Zbudowany program  umożliwia wykonanie operacji splotu, korelacji bezpośredniej oraz korelacji na podstawie splotu. W programie możliwa jest również filtracja dolnoprzepustowa oraz środkowoprzepustowa z możliwością wyboru okna prostokątnego bądź Hamminga. 

Wykonanie operacji korelacji powoduje minimalne przesunięcie się wykresu w prawą stronę. Dla dwóch sygnałów sinusoidalnych wykonanie operacji korelacji daje wykres będący odwórconym wykresem operacji splotu wzdłuż osi poziomej przechodzącej przez środek wykresu. Natomiast dla różnych sygnałów - sygnału trójkątnego oraz sygnału sinusoidalnego wyprostowanego jednopołówkowo, a także dla sygnału prostokątnego oraz szumu gaussowskiego powoduje przesunięcie wykresu w sotsunku do wyrkesu splotu na tych smaych wykresach o minimalną odległość w prawo. Nie zauważyliśmy różnicy pomiędzy wykresami dla operacji korelacji bezpośredniej oraz wykresami dla korelacji z użyciem splotu. Wykresy operacji splotu oraz operacji korelacji są do siebie bardzo zbliżone.  

Filtracja z oknem Hamminga daje wykres bardziej spłaszczony dla wartości bliskich zeru w porównaniu do wykresu filtracji z oknem prostokątnym. Wartości maksymalna oraz minimalna są takie same. 

Opisane wnioski dla okna prostokątnego oraz okna Hamminga są identyczne dla filtracji dolnoprzepustowej oraz filtracji środkowoprzepustowej. 

Natomiast porównując wykresy filtracji dolnoprzepustowej i środkowoprzepustowej zauważamy, że widoczna zmiana następuje w warotści minimalnej dla wykresu co powoduje, że dla wykresu filtracji dolnorprzepustowej punkty skupiające się przy wartości 0 znajdują sie na dole wykresu, natomaist dla wykresu filtracji środkowoprzepustowej znajdują się one na środku wykresu.

Opisane wnioski dla filtracji dolnoprzepustowej oraz środkowoprzepustowej są identyczne dla okna prostokątnego oraz okna Hamminga. 


\begin{thebibliography}{0}
\bibitem{dane}  Wikamp, Instrukcja do zadania trzeciego, Dostępny w: \url{https://ftims.edu.p.lodz.pl/pluginfile.php/14039/mod_resource/content/1/zad3.pdf}

\end{thebibliography}



\end{document}