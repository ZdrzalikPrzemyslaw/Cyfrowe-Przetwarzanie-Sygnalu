\documentclass[12pt]{article}
\usepackage[T1]{fontenc}
\usepackage[T1]{polski}
\usepackage[utf8]{inputenc}
\newcommand{\BibTeX}{{\sc Bib}\TeX} 
\usepackage{graphicx}
\usepackage{amsfonts}
\usepackage{url}
\usepackage{babel,blindtext}
\usepackage{hyperref}


\setlength{\textheight}{21cm}

\title{{\bf Zadanie nr 4 - Przekształcenie Fouriera, Walsha-Hadamarda, kosinusowe i falkowe, szybkie algorytmy.}\linebreak
Cyfrowe Przetwarzanie Sygnałów}
\author{Julia Szymańska, 224441\and Przemysław Zdrzalik, 224466}
\date{17.06.2021r.}

\begin{document}
\clearpage\maketitle
\thispagestyle{empty}
\newpage
\setcounter{page}{1}


%%%%%%%%%%%%%%%%%%%%%%%%%%%%%%%%%%%%%%%%%%%%%%%%%%%%%%%%%%%%%%%%%%%%%%%%%%%%%

\section{Cel zadania}
Celem ćwiczenia jest budowa programu umożliwiającego wykonanie operacji transformacji sygnałów dyskretnych przy użyciu wybranych metod.

%%%%%%%%%%%%%%%%%%%%%%%%%%%%%%%%%%%%%%%%%%%%%%%%%%%%%%%%%%%%%%%%%%%%%%%%%%%%%

\section{Wstęp teoretyczny}

 Do operacji transformacji w programie zostały wykorzystane wzory znajdujące się w instrukcji do zadania czwartego na platformie Wikamp \cite{dane}. 


%%%%%%%%%%%%%%%%%%%%%%%%%%%%%%%%%%%%%%%%%%%%%%%%%%%%%%%%%%%%%%%%%%%%%%%%%%%%%

\section{Eksperymenty i wyniki}

Do wykonania transformacji zostały użyte sygnały określone wzorem:

\begin{figure}[h!]
 \centering
 \includegraphics[width=12cm]{wzory.png}
 \vspace{-0.3cm}
 \caption{Wzory sygnałów S1, S2, S3, użytych w poniższych eksperymentach. }
 \label{rysunek do eksperymentu 1 wariantu 1}
\end{figure}


\begin{figure}[h!]
 \centering
 \includegraphics[width=8cm]{s1.png}
 \vspace{-0.3cm}
 \caption{Wykresy sygnału S1.}
 \label{rysunek do eksperymentu 1 wariantu 1}
\end{figure}
\newpage

\begin{figure}[h!]
 \centering
 \includegraphics[width=8cm]{s2.png}
 \vspace{-0.3cm}
 \caption{Wykresy sygnału S2.}
 \label{rysunek do eksperymentu 1 wariantu 1}
\end{figure}

\begin{figure}[h!]
 \centering
 \includegraphics[width=8cm]{s3.png}
 \vspace{-0.3cm}
 \caption{Wykresy sygnału S3.}
 \label{rysunek do eksperymentu 1 wariantu 1}
\end{figure}
\newpage

%%%%%%%%%%%%%%%%%%%%%%%%%%%%%%%%%%%%%%%%%

\subsection{Eksperyment nr 1 - Porównanie DTF oraz FFT2 (przerzedzenie w częstotliwości)}

\begin{figure}[h!]
 \centering
 \includegraphics[width=10cm]{E1_S1_dft.png}
 \vspace{-0.3cm}
 \caption{Wykresy części rzeczywistej i urojonej dla liczb zespolonych będących wynikiem transformacji DTF oraz FFT2 dla S1.}
 \label{rysunek do eksperymentu 1 wariantu 1}
\end{figure}

\begin{figure}[h!]
 \centering
 \includegraphics[width=10cm]{E1_S1_dft2.png}
 \vspace{-0.3cm}
 \caption{Wykresy modułu i arguemntu w dziedzinie częstotliwości dla liczb zespolonych będących wynikiem transformacji DTF oraz FFT2 dla S1.}
 \label{rysunek do eksperymentu 1 wariantu 1}
\end{figure}
\newpage

\begin{table}[h!]
\caption{Porównanie czasu wykonywania transformacji DFT i FFT2 w milisekundach dla S1.}
\centering
\vspace{0.1cm}
 \begin{tabular}{c c}
    \textbf{DFT} & \textbf{FFT2}  \\
\hline
3,997 & 4,038 \\
\end {tabular}
\label {Parametry wejściowe dla trzeciego eksperymetnu. }
\end{table}

%%%%
\begin{figure}[h!]
 \centering
 \includegraphics[width=10cm]{E1_S2_dft.png}
 \vspace{-0.3cm}
 \caption{Wykresy części rzeczywistej i urojonej dla liczb zespolonych będących wynikiem transformacji DTF oraz FFT2 dla S2.}
 \label{rysunek do eksperymentu 1 wariantu 1}
\end{figure}
\newpage

\begin{figure}[h!]
 \centering
 \includegraphics[width=10cm]{E1_S2_dft2.png}
 \vspace{-0.3cm}
 \caption{Wykresy modułu i arguemntu w dziedzinie częstotliwości dla liczb zespolonych będących wynikiem transformacji DTF oraz FFT2 dla S2.}
 \label{rysunek do eksperymentu 1 wariantu 1}
\end{figure}


\begin{table}[h!]
\caption{Porównanie czasu wykonywania transformacji DFT i FFT2 w milisekundach dla S2.}
\centering
\vspace{0.1cm}
 \begin{tabular}{c c}
    \textbf{DFT} & \textbf{FFT2}  \\
\hline
4,037 & 3,999 \\
\end {tabular}
\label {Parametry wejściowe dla trzeciego eksperymetnu. }
\end{table}
\newpage

%%%%
\begin{figure}[h!]
 \centering
 \includegraphics[width=10cm]{E1_S3_dft.png}
 \vspace{-0.3cm}
 \caption{Wykresy części rzeczywistej i urojonej dla liczb zespolonych będących wynikiem transformacji DTF oraz FFT2 dla S3.}
 \label{rysunek do eksperymentu 1 wariantu 1}
\end{figure}

\begin{figure}[h!]
 \centering
 \includegraphics[width=10cm]{E1_S3_dft2.png}
 \vspace{-0.3cm}
 \caption{Wykresy modułu i arguemntu w dziedzinie częstotliwości dla liczb zespolonych będących wynikiem transformacji DTF oraz FFT2 dla S3.}
 \label{rysunek do eksperymentu 1 wariantu 1}
\end{figure}
\newpage

\begin{table}[h!]
\caption{Porównanie czasu wykonywania transformacji DFT i FFT2 w milisekundach dla S3.}
\centering
\vspace{0.1cm}
 \begin{tabular}{c c}
    \textbf{DFT} & \textbf{FFT2}  \\
\hline
4,001 & 7,037 \\
\end {tabular}
\label {Parametry wejściowe dla trzeciego eksperymetnu. }
\end{table}



%%%%%%%%%%%%%%%%%%%%%%%%%%%%%%%%%%%%%%%%%

\subsection{Eksperyment nr 2 - Transformacja kosinusowa}

\begin{figure}[h!]
 \centering
 \includegraphics[width=10cm]{E2_S1.png}
 \vspace{-0.3cm}
 \caption{Wykres transformacji kosinusowej dla S1.}
 \label{rysunek do eksperymentu 1 wariantu 1}
\end{figure}
\newpage

\begin{figure}[h!]
 \centering
 \includegraphics[width=10cm]{E2_S1_s.png}
 \vspace{-0.3cm}
 \caption{Wykres szybkiej transformacji kosinusowej dla S1.}
 \label{rysunek do eksperymentu 1 wariantu 1}
\end{figure}


\begin{figure}[h!]
 \centering
 \includegraphics[width=10cm]{E2_S1_o.png}
 \vspace{-0.3cm}
 \caption{Wykres odwrotnej transformacji kosinusowej dla S1.}
 \label{rysunek do eksperymentu 1 wariantu 1}
\end{figure}

\begin{table}[h!]
\caption{Porównanie czasu wykonywania transformacji kosinusowej i szybciej transformacji kosinusowej w milisekundach dla S1.}
\centering
\vspace{0.1cm}
 \begin{tabular}{c c}
    \textbf{kosinusowa} & \textbf{szybka kosinusowa}  \\
\hline
4,987 & 2,990 \\
\end {tabular}
\label {Parametry wejściowe dla trzeciego eksperymetnu. }
\end{table}
\newpage


%%%%
\begin{figure}[h!]
 \centering
 \includegraphics[width=10cm]{E2_S2.png}
 \vspace{-0.3cm}
 \caption{Wykres transformacji kosinusowej dla S2.}
 \label{rysunek do eksperymentu 1 wariantu 1}
\end{figure}


\begin{figure}[h!]
 \centering
 \includegraphics[width=10cm]{E2_S2_s.png}
 \vspace{-0.3cm}
 \caption{Wykres szybkiej transformacji kosinusowej dla S2.}
 \label{rysunek do eksperymentu 1 wariantu 1}
\end{figure}
\newpage

\begin{figure}[h!]
 \centering
 \includegraphics[width=10cm]{E2_S2_o.png}
 \vspace{-0.3cm}
 \caption{Wykres odwrotnej transformacji kosinusowej dla S2.}
 \label{rysunek do eksperymentu 1 wariantu 1}
\end{figure}

\begin{table}[h!]
\caption{Porównanie czasu wykonywania transformacji kosinusowej i szybciej transformacji kosinusowej w milisekundach dla S2.}
\centering
\vspace{0.1cm}
 \begin{tabular}{c c}
    \textbf{kosinusowa} & \textbf{szybka kosinusowa}  \\
\hline
5,982 & 1,994 \\
\end {tabular}
\label {Parametry wejściowe dla trzeciego eksperymetnu. }
\end{table}
\newpage

%%%%
\begin{figure}[h!]
 \centering
 \includegraphics[width=10cm]{E2_S3.png}
 \vspace{-0.3cm}
 \caption{Wykres transformacji kosinusowej dla S3.}
 \label{rysunek do eksperymentu 1 wariantu 1}
\end{figure}


\begin{figure}[h!]
 \centering
 \includegraphics[width=10cm]{E2_S3_s.png}
 \vspace{-0.3cm}
 \caption{Wykres szybkiej transformacji kosinusowej dla S3.}
 \label{rysunek do eksperymentu 1 wariantu 1}
\end{figure}
\newpage


\begin{figure}[h!]
 \centering
 \includegraphics[width=10cm]{E2_S3_o.png}
 \vspace{-0.3cm}
 \caption{Wykres odwrotnej transformacji kosinusowej dla S3.}
 \label{rysunek do eksperymentu 1 wariantu 1}
\end{figure}


\begin{table}[h!]
\caption{Porównanie czasu wykonywania transformacji kosinusowej i szybciej transformacji kosinusowej w milisekundach dla S3.}
\centering
\vspace{0.1cm}
 \begin{tabular}{c c}
    \textbf{kosinusowa} & \textbf{szybka kosinusowa}  \\
\hline
2,990 & 2,992 \\
\end {tabular}
\label {Parametry wejściowe dla trzeciego eksperymetnu. }
\end{table}
\newpage

%%%%%%%%%%%%%%%%%%%%%%%%%%%%%%%%%%%%%%%%%%%%%%%%%%%%%%%%%%%%%%%%%%%%%%%%%%%%%
\newpage
\section{Wnioski}


\begin{thebibliography}{0}
\bibitem{dane}  Wikamp, Instrukcja do zadania trzeciego, Dostępny w: \url{https://ftims.edu.p.lodz.pl/pluginfile.php/14303/mod_resource/content/0/zadanie4.pdf}

\end{thebibliography}



\end{document}